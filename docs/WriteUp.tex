
\documentclass{article}

\usepackage{hyperref}
\usepackage{tabularx}
\usepackage{gantt}
\usepackage{listings}
\usepackage{color}
\usepackage{float}


\hypersetup{
    colorlinks=true, %set true if you want colored links
    linktoc=all,     %set to all if you want both sections and subsections linked
    linkcolor=black,  %choose some color if you want links to stand out
}

\title{EPQ Writeup}
\date{}
\author{James Rand}

\def\tightlist{}

\makeatletter
\def\@seccntformat#1{%
    \expandafter\ifx\csname c@#1\endcsname\c@section\else
    \csname the#1\endcsname\quad
\fi}
\makeatother


\definecolor{mygreen}{rgb}{0,0.6,0}
\definecolor{mygray}{rgb}{0.5,0.5,0.5}
\definecolor{mymauve}{rgb}{0.58,0,0.82}

\lstset{ %
    backgroundcolor=\color{white},   % choose the background color; you must add \usepackage{color} or \usepackage{xcolor}; should come as last argument
    basicstyle=\footnotesize,        % the size of the fonts that are used for the code
    breakatwhitespace=false,         % sets if automatic breaks should only happen at whitespace
    breaklines=true,                 % sets automatic line breaking
    captionpos=b,                    % sets the caption-position to bottom
    commentstyle=\color{mygreen},    % comment style
    deletekeywords={...},            % if you want to delete keywords from the given language
    escapeinside={\%*}{*)},          % if you want to add LaTeX within your code
    extendedchars=true,              % lets you use non-ASCII characters; for 8-bits encodings only, does not work with UTF-8 frame=single,                    % adds a frame around the code keepspaces=true,                 % keeps spaces in text, useful for keeping indentation of code (possibly needs columns=flexible) keywordstyle=\color{blue},       % keyword style language=C,                      % the language of the code
    morekeywords={*,...},            % if you want to add more keywords to the set
    numbers=left,                    % where to put the line-numbers; possible values are (none, left, right)
    numbersep=5pt,                   % how far the line-numbers are from the code
    numberstyle=\tiny\color{mygray}, % the style that is used for the line-numbers
    rulecolor=\color{black},         % if not set, the frame-color may be changed on line-breaks within not-black text (e.g. comments (green here))
    showspaces=false,                % show spaces everywhere adding particular underscores; it overrides 'showstringspaces'
    showstringspaces=false,          % underline spaces within strings only
    showtabs=false,                  % show tabs within strings adding particular underscores
    stepnumber=2,                    % the step between two line-numbers. If it's 1, each line will be numbered
    stringstyle=\color{mymauve},     % string literal style
    tabsize=2,                       % sets default tabsize to 2 spaces
    title=\lstname                   % show the filename of files included with \lstinputlisting; also try caption instead of title
}

\begin{document}


\maketitle

\tableofcontents
\setcounter{tocdepth}{5}

\newpage

\section{Introduction}
\subsection{Objective}

In this extended project I plan to design a device that will be able to interact with a file
server. This device must be able to achieve some of the following functions:

\begin{itemize}
    \item Be able to turn the server on and off
    \item Be power efficient so the running cost is low
    \item forward a shell connection onto the server
    \item Have an interface to interact with the device
    \item Do not exceed the budget of \$50
\end{itemize}

The first object that I wish to discuss is having the ability to remotely turn on and off the
server. This could be done in two ways: the first would be to use wake on LAN which would be
a way of doing it through software; the second is using electrical components to act as an
electric switch. \\

The first approach I would like to talk about is using wake on LAN. This seems to be the default
way of doing what I am proposing which means that setting this up will be straight forward and
well documented. The only problem with using this method is that the server will constantly need
to be on standby which conflicts with my second goal of this project being power efficient. The
second approach is the opposite of the first as it will be more difficult to set up due to the
fact it will require me to buy parts which need to be appropriate to interact the server. On the
plus side though it could be a more power efficient as I could design it to control the mains
power supply as well so that it will not need to be on standby.

\subsection{Time scale}

Since I have a limited amount of time to achieve my goal I have created the following Gantt
chart (Fig ~\ref{fig:Gantt Chart}) to illustrate how I intend to use my time. This chart
only shows how I will use the time up to the initial draft. There is time available after
this draft but I would like to use this time to improve my write up and act upon feedback.

\begin{figure}
    \begin{gantt}[xunitlength=\textwidth / 16]{7}{16}
        \begin{ganttitle}
            \titleelement{Number of weeks}{16}
        \end{ganttitle}
        \begin{ganttitle}
            \numtitle{1}{1}{16}{1}
        \end{ganttitle}
        \ganttbar{Research}{0}{7}
        \ganttbarcon{Design}{7}{2}
        \ganttbarcon{Building}{9}{4}
        \ganttbarcon{Reflection}{13}{3}
        \ganttmilestone[color=cyan]{Draft deadline}{16}
        %\ganttmilestonecon{A connected Milestone}{7}
    \end{gantt}
    \caption{Gantt chart} \label{fig:Gantt Chart}
\end{figure}

\break
\section{Research}
\subsection{choosing the right board}

The very first thing that I need to consider when designing this is what development board
I will as everything else will be dependant upon this. The 4 factors that I decided would
be most important to consider are: the price of the board; how easy it would to connect
and control through the internet and power consumption.

The reason I think power consumption should be something to consider is that this board will
be on for long periods of time to make sure that the server is always available. Which means
running a power hungry device will be expensive and not practical. The price is an obvious
important feature and since I may be buying extra hardware components I need to be extra wary of
the cost of the board. Since I will be controlling the board through the internet it needs to
have internet connectivity and be straightforward to setup. Finally I need to think about the
implications of the hardware. This means the board must be able to communicate to hardware from
a custom written program.

\subsubsection{Raspberry Pi}

The Raspberry Pi is definitely one of the more well known development boards that are use
extensively by hobbyists. This means that the Raspberry Pi will be well documented so if
I come across any difficulties it will be easier to troubleshoot. The table below compares
the technical specs of all of the models. \\

\begin{tabularx}{\textwidth}{| X | X | X | X | X |}
    \hline
    Raspberry Pi Model & Model A+ & Model B & Model B+ & Model B 2 \\ \hline
    Price              &  \$19.99  &  \$39.99 &  \$29.99  &  \$39.99   \\ \hline
    Processor Speed    &  \multicolumn{4}{c |}{700 Mhz}  \\ \hline
    Power Consumption  &  \multicolumn{3}{c |}{600mA @ 5V} & 650mA @ 5V \\ \hline
    Memory             &  256MB   & \multicolumn{2}{c |}{512MB} & 1GB \\ \hline
    Storage            &  Micro SD Card &  SD Card & \multicolumn{2}{c |}{Micro SD Card} \\ \hline
    Pins               &  40      &    26  & \multicolumn{2}{c |}{40} \\ \hline
    Coding Languages   &  \multicolumn{4}{c |}{Python} \\ \hline
\end{tabularx}
\newline

The source of the information is in the refrences\cite{raspberryPi}
\newline

The characteristics that I have mentioned above make the raspberry effortless to develop on;
mainly because it runs the Linux operating system. This means that I can use and run any language
that runs Linux giving me a wide range of tools to use. I could also install and use programs that
other people have written. As mention above the raspberry pi is very popular meaning the is a wide
range of hardware extension that may make it easier to control external components.

\subsubsection{BeagleBone Black / BeagleBone Black Wireless}

This section will cover a comparison of features on the BeagleBone Black and the BeagleBone Black Wireless.
The only difference between these boards is that one has bluetooth and wifi while the other has ethernet.
These boards, similar to the raspberry pi, are very popular meaning there is a wide range of documentation
and add-ons that can be used although it should be noted that they are not as popular as the raspberry pi.
The table below will give a summery of the features available. \\

%\newline
\begin{tabularx}{\textwidth}{| X | X | X |}
    \hline
    BeagleBone Model &  BeagleBone Black  & BeagleBone Black Wireless \\ \hline
    Price            &       \$75         &        \$80               \\ \hline
    Processor Speed  &  \multicolumn{2}{c |}{1GHz}                    \\ \hline
    Power Consumption&  \multicolumn{2}{c |}{210-460mA @ 5V}          \\ \hline
    Memory           &  \multicolumn{2}{c |}{512MB}                   \\ \hline
    Storage          &  \multicolumn{2}{c |}{2GB eMMC}                \\ \hline
    Pins             &  \multicolumn{2}{c |}{2x46 pin headers}        \\ \hline
    Coding Languages &  \multicolumn{2}{c |}{C/C}                   \\ \hline
\end{tabularx}
\newline

The source of the information is in the refrences\cite{beagleboard}
\newline

The first thing that I noticed when researching this board is that it can be pricey but looking
at the specs you can see why. These boards are very similar to the raspberry pi but I thought it
was worth discussing this board as it excels in hardware thanks to it's many pins. Similarly to the
raspberry pi it also runs Linux. The BeagleBone also has lower power consumption than the raspberry
pi. Finally it is worth noting that it uses C/C++ for the programming language which is more difficult
to write in than a language such as Python.

\subsubsection{Arduino Uno}

The arduino is a classic development board that instead of the previous two items does not have the Linux
operating system. That being said due to it's large popularity there are a huge number of libraries and
APIs that you can use. \\

\begin{tabularx}{\textwidth}{| X | X |}
    \hline
    Arduino Model    &   Arduino Uno \\ \hline
    Price            &       \$25    \\ \hline
    Processor Speed  &      16MHz    \\ \hline
    Power Consumption&    50mA @ 5v  \\ \hline
    Memory           &      32KB     \\ \hline
    Pins             &       20      \\ \hline
    Coding Languages &     C/C++     \\ \hline
\end{tabularx}
\newline

The source of the information is in the refrences\cite{ardino}
\newline

The specs on the arduino are considerably lower than the previous two items due to the fact that
the arduino is a microcontroller instead of a size down computer. This means that it consumes less
power and has less features that the other two can offer. There is also a problem with wireless
capabilities on the arduino as it has none built in and would require me to buy a WiFi sheild.

\subsubsection{Particle Photon}

This board is much more obscure than the rest of the boards that I have compared. Despite this there
are a lot of attractive characteristics that this board like the fact that it has in built wireless
and can be communicated with through the internet without the need to port forward. Additionally this
board is the smallest of all of the ones that I have considered. \\

\begin{tabularx}{\textwidth}{| X | X |}
    \hline
    Particle Model   &    Photon     \\ \hline
    Price            &     \$19      \\ \hline
    Processor Speed  &    120Mhz     \\ \hline
    Power Consumption& 80mA @ 5V-3.3V\\ \hline
    Memory           &    128KB      \\ \hline
    Storage          &     1MB       \\ \hline
    Pins             &     28        \\ \hline
    Coding Languages &    C/C++      \\ \hline
\end{tabularx}
\newline

The source of the information is in the refrences\cite{particlePhotonSpec}
\newline

Similarly to the Arduino this board is a microcontroller. The features I mentioned above regarding
connectivity will make development considerable easier. This is also the cheapest board that I have
compared giving me plenty of left over money to buy all the hardware components that I might need.


\subsubsection{The Board I Chose}

After considering the pros and cons of all of these boards I decided that a microcontroller type
would be much more suitable for my project. My reasoning behind this is that a fast processor and
a Linux operating system is unnecessary in this situation meaning that I would be wasting power on
features that I am not going to use. The price is also higher for the non-micro-controllers.

I now needed to make a decision between the Arduino and the Particle Photon. Although the arduino has
a huge number of libraries and a comprehensive documentation; I came to the conclusion

\subsection{Transistors vs Relays}

In order to interact with the server on a hardware level I need a component that can act like an
electrical switch. This is so that when I need to turn the server on and off I can achieve this with
this "electrical switch". There are two components that took my interest while researching this. The
first is transistors which are part of the semiconductor family and are largely used in computers. The
second is a relay which unlike a transistor has moving parts and works using an electro magnet. The first
that I will discuss in detail are transistors.

\subsubsection{Transistors}

\begin{figure}
    There are many different types of transistors but the most common and the one I am most interested in is
    the "Bipolar junction transistor" or "BJT" for short. There are two types of BJT that I need to discuss
    which are NPN and PNP. Before I start describing those it is important to understand the basic anatomy
    of transistors first.

    \includegraphics{pictures/transistor-layout.png}
    \caption{transistor layout} \label{fig:transistor-layout}
    \vspace{0.5cm}

    In the layout above (Fig ~\ref{fig:transistor-layout}) you can clearly see that a transistor has three legs.
    These legs each do different things and have special names: collector, base and emitter. The leg in the middle
    called the base is the one we are most interested in as this can control the current through the emitter and
    collector by varying the input to it. The collector and emitter can change based upon the type of transistor

\end{figure}

\begin{figure}
    \includegraphics{pictures/npn-pnp-symbols.png}
    \caption{NPN vs PNP} \label{fig:different-transitor-layout}
    \vspace{0.5cm}

    The two different diagrams above show the different types of BJT transistors
    (Fig ~\ref{fig:different-transistor-layout}). \\

    The only difference between these two types which is illustrated by this figure is the direction of the
    emitter arrow. This small difference has big implications and lead to different properties for each. I
    will now start to discuss these consequences and which out of the two I think is most suitable for this
    project.

\end{figure}

\break

Unlike many other electrical components like resistors and capacitors which have a linear relationship with
voltage and current transistors have four different ways of handling electrical power:

\begin{itemize}
    \item Saturation     --- Which freely allows voltage and current to flow from the collector to emitter
    \item Cut-off        --- Allows no voltage or current from collector to emitter. Opposite of "Saturation"
    \item Active         --- The current from the collector to emitter is proportional to the current flowing
        through the base
    \item Reverse-Active --- Similar to "Active" the current is proportional to the base but flows in reverse
\end{itemize}

For the purposes of this project the only modes worth discussing are "Saturation" and "Cut-off" because we
are only interested in on or off and not anything in-between.

To determine what mode the transistor is in you need to investigate what voltage is supplied to each pin.
In order to put a NPN transistor into saturation mode the voltage supplied to the base, V\textsubscript{B}
needs to be greater that the voltage supplied to the emitter, V\textsubscript{E}. To put into cut-off mode
it needs to be the opposite. Likewise with PNP transistors it is reversed.

After considering these factors I decided to perform an experiment to get use to using transistors. I did
this by connect up an led and transistor to my board and checking that I can turn it off and on by emitting
an output voltage on one of the pins. A schematic of my experiment can be seen below in figure >>>>>>.



I flash a example program onto the board called "Blinky". The purpose of this program to to turn a output
pin on and off every second. This output pin is labelled the voltage alternator in the diagram. After I
design this circuit I then built it on a bread board which is shown below in figure >>>>>>>. The result
of this is the led turn off and the on again.

The source of the information for this section can be found in the references \cite{transistor}

\subsubsection{Relay}

An alternative to the transistors is the relay. Similarly to the transistor this can work as an electronic
switch. The main difference between a relay and a transistor is that a relay have physical moving parts
inside. Inside a relay there is an electro magnet which can pull a thin piece of metal down to make the
switch close. This is how a relay works. Due to this mechanism a relay can be significantly bigger than
a transistor and it can also handle higher voltages with ease. A downside to a relay is that it can
take longer for it to switch on and off. Luckily with my choice of board there is a add-on board that has
relay support called the "Relay shield"\cite{relaySheild}. This make setting up the relays significantly easier than setting
up the transistors. This relay shield can also handle high voltages meaning it may be able to control the
mains supply to the server making it more power efficient than transistors. The only downside to using this
add-on is that it is significantly more expensive than using the raw components. A picture of this
add-on board is shown down below in figure ~\ref{fig:relayShield}.


\begin{figure}[H]
    \noindent\makebox[\textwidth]{\includegraphics[width=\paperwidth-3cm]{pictures/relay-sheidl.jpg}};
    \caption{Picture of the relay shield} \label{fig:relayShield}
\end{figure}

\subsubsection{Conclusion}
In conclusion I decided to go with the relay add-on board due to it being design for this exact practice and
it's convenient. Even though the experiment performed with the transistor was successful it would fail on
higher voltages on the mains therefore it would be less power efficient.

\section{Design}

After doing this research I think it is now time to move on and start building schematics and pseudo code
for how I want the project to look like. I think the most critical component of this project is the
actual functionality of turning the relay on and off remotely.

\subsection{Turning the relay off and on remotely}
I think the first thing I would like to approach would be the software behind this action. My reasoning
behind this is that the hardware set-up will be very similar to the relay test hardware the only difference
is instead of having an led I will have the two power wires from the server.

\subsubsection{Software}
Before describing how I plan to turn the server off and on I would like to explain the basic structure of
the particle photon api.

Within the particle photon api there are three basic functions: set-up() and loop(). The first function
set-up is run when the particle photon boots up so this should be use to initialize any important variables
that will be used throughout the function. The second procedure loop repeats all of the code inside as long
as the photon is switch on. Below is some example code that was use during the tests for the relay and the
transistor called blinky. This simply turns an pin output on and off and is show in Fig ~\ref{fig:exampleCode}

\begin{figure}
    \begin{lstlisting}

        int led1 = D0; // Instead of writing D0 over and over again, we'll write led1
        int led2 = D7; // Instead of writing D7 over and over again, we'll write led2

        void setup() {

          // We are going to tell our device that D0 and D7 (which we named led1 and led2 respectively) are going to be output
          // (That means that we will be sending voltage to them, rather than monitoring voltage that comes from them)

          // It's important you do this here, inside the setup() function rather than outside it or in the loop function.

          pinMode(led1, OUTPUT);
          pinMode(led2, OUTPUT);

        }

        // Next we have the loop function, the other essential part of a microcontroller program.
        // This routine gets repeated over and over, as quickly as possible and as many times as possible, after the setup function is called.
        // Note: Code that blocks for too long (like more than 5 seconds), can make weird things happen (like dropping the network connection).  The built-in delay function shown below safely interleaves required background activity, so arbitrarily long delays can safely be done if you need them.

        void loop() {
          // To blink the LED, first we'll turn it on...
          digitalWrite(led1, HIGH);
          digitalWrite(led2, HIGH);

          // We'll leave it on for 1 second...
          delay(1000);

          // Then we'll turn it off...
          digitalWrite(led1, LOW);
          digitalWrite(led2, LOW);

          // Wait 1 second...
          delay(1000);

          // And repeat!
        }
    \end{lstlisting}
    \caption{Example Code} \label{fig:ExampleCode}
    \vspace{0.5cm}
    The comment lines denoted by "//" explain what is happening in the code.
\end{figure}

Now that I have explained the basics I can now describe some code that should allow me to turn
a relay on and off. This code is shown in ~\ref{fig:relayTest}.
\begin{figure}
    \begin{lstlisting}
        int relayState = 0;


        void setup()
        {
           //Initilize the relay control pins as output
           pinMode(RELAY1, OUTPUT);
           // Initialize all relays to an OFF state
           digitalWrite(RELAY1, LOW);
           //register the Particle function
           Particle.function("relay", relayControl);
        }

        void loop()
        {
           // This loops for ever
        }

        // command format r1,HIGH
        int relayControl(String command)
        {
          // invert the relay number
            relayState = !relayState;


          // write to the appropriate relay
          digitalWrite(0, relayState);
          return 1;
        }
    \end{lstlisting}
    \caption{Relay Test Code} \label{fig:relayTest}
    \vspace{0.5cm}
    The code is similar to the previous example (Fig ~\ref{fig:ExampleCode}) the only big
    difference in the "relayControl" function. All this function does is turn the relayState
    variable to a one if it is a zero and to a zero if it is a one. Once it has done this it
    set the relay to this new state. In the setup() function I register this function with
    particle which means that I can now trigger this function remotely using a POST request.
\end{figure}

\pagebreak
To trigger this function with a http POST request I needed to know two pieces of information
. The first is my device id which is a number that is assign to every particle photon connect
to the particle cloud. This is use so the particle cloud know what device I want to select.
The second piece of information is my access token. This another number that is assign to
my account instead of the actual device. This enables the particle cloud to verify that I
am who I say I am. Similar to a password.

I can then modify this code to get to be more specific to the turning on the server.

\begin{figure}
    \begin{lstlisting}
    int RELAY1 = D0;

    void setup()
    {
       //Initilize the relay control pins as output
       pinMode(RELAY1, OUTPUT);
       // Initialize all relays to an OFF state
       digitalWrite(RELAY1, LOW);

       //register the Particle function
       Particle.function("relay", relayControl);
    }

    void loop()
    {
       // This loops for ever
    }

    // command format r1,HIGH
    int relayControl(String command)
    {
        digitalWrite(0, 1);
        delay(500);
        digitalWrite(0, 0);

        return 1;
    }
    \end{lstlisting}
    \caption{Relay Code} \label{fig:relayCode}
    \vspace{0.5cm}
    This code turns on the relay for 500 milliseconds and then turns it back on again
    meaning the relayControl function is now used exclusively to turn on the server.
    This is because later on I find an interest feature of the server meaning the photon
    will not be turning the server off.

\end{figure}

\subsubsection{The Hardware}
The next thing to design in terms of turning on the server is to look at the hardware.
Fortunately it is simple to set up and is similar to the hardware in the relay test. The main
part of the hardware are two relays that control mains power supply and turning the server on
from standby and are connected up to the pins "D1" and "D0" respectively. I have also added a
override switch which will block any attempt to turn on the server. There is however a difference
between the relay responsible for the mains power supply and the relay for turning on the server
and that is the inputs that they are connected up to. The mains power supply is hooked up to the NC
input meaning that current can flow through normally unless the particle photon powers the relay. While
the relay for turning on the server is connected to NO input meaning that current can only flow
if the relay is powered. This makes sense as in the next section I talk about automatically turning
off the relay board after it is finished turning on the server meaning that it will not be able to
consistently power on the relay on the mains.

\begin{figure}[H]
    \noindent\makebox[\textwidth]{\includegraphics[width=\paperwidth-3cm]{pictures/turnOnRelay.png}};
    \caption{Hardware schematic} \label{fig:turnOnRelay}
\end{figure}


\subsubsection{Taking advantage of a quirk of the server}
Interestingly the server exhibits an odd behaviour that I can take advantage of to improve my
project which is that the usb ports are not powered when the server is on standby but is powered
when it is turn on. This is useful for two reasons: the first is that I can use it to properly
check whether the server has successfully turn on and the second is that I can use this new
power to turn off the particle photon board after it has finished turning on the server.

This does have a side affect however. Which is that I can no longer use the particle
photon to turn off the server. This is easily resolved as I plan to have an SSH interface
with the server meaning I turn it off through that connection.

This new feature requires me to add to the design of the hardware. Figure ~\ref{fig:usbSch}
essentially outlines what I want to add. It shows a relay hooked up to the power source of
the photon so that it is normally closed. But when power is supplied to this relay from the
usb the relay becomes open turning off the board.

\begin{figure}[H]
    \noindent\makebox[\textwidth]{\includegraphics[width=\paperwidth-3cm]{pictures/usbLoopSch.png}};
    \caption{Usb schematic} \label{fig:usbSch}
\end{figure}

\subsection{Forwarding a shell}
Another requirement for device is to be able to forward a shell onto the server. Doing this
purely from this device is counter productive and goes beyond the scope of this project. The
approach that make most sense for this project is to have the particle photon grab the public
IP address and return this to a client whenever the client issues the command to turn on the
server. This client will be the the interface for the device stated in one of the predetermined
goals. For the moment I will analyse what is need to grab the public IP on the device. Fortunately
this can be done using the particle photon api\cite{publicIPDocs}. The example code given by the
documentation is given below in figure
~\ref{fig:publicIpExample}

\begin{figure}[H]
    \begin{lstlisting}
    // Open a serial terminal and see the IP address printed out
    void handler(const char *topic, const char *data) {
        Serial.println("received " + String(topic) + ": " + String(data));
    }

    void setup() {
        Serial.begin(115200);
        Particle.subscribe("particle/device/ip", handler);
        Particle.publish("particle/device/ip");
    }
    \end{lstlisting}
    \caption{Public Ip Example code} \label{fig:publicIpExample}
    %\vspace{0.5cm}
    The two main things to focus here are the particle subscribe and publish the rest of the
    code is to set up and display the ip every time it changes on a serial screen. Since
    I do not have a serial connection and nor do I want one I shall not clarify what it
    is or how it works.
\end{figure}

The "Particle.subscribe" function allows the user to listen out for certain events and then
attach a custom to these event so that every time the event occurs the custom function is
called. This is very useful for this project as  public ip address are subject to change meaning
if I use this as the ip address will always be the current one. The "Particle.publish" is a way
of manually triggering the event specified in the parameters. These two lines can be used
in the device code where the we can attach a custom function that changes a global string variable
that contains the ip address. Then it's easy to change the relay control function to return this
variable instead of an integer. Once I have this value it should be easy to create a SSH session
with interface and the server. I think this is better than trying to create my own connect as it
will be more secure and easier to set up.

With the code for the relay (Fig ~\ref{fig:relayCode}) mix in with this additional code it should look
something like this:

\begin{figure}[H]
    \begin{lstlisting}
    int RELAY1 = D0;
    char ip[15];

    void setup()
    {
       //Initilize the relay control pins as output
       pinMode(RELAY1, OUTPUT);
       // Initialize all relays to an OFF state
       digitalWrite(RELAY1, LOW);

       //register the Particle function
       Particle.function("relay", relayControl);
       Particle.subscribe("particle/device/ip" , handler);
       Particle.publish("particle/device/ip");

    }

    void loop()
    {
       // This loops for ever
    }

    void handler(const char *topic, const char *data)
    {
        //sprintf(ip, "%s", data);
        strcpy(ip, data);
    }

    // command format r1,HIGH
    String relayControl(String command)
    {
        digitalWrite(0, 1);
        delay(500);
        digitalWrite(0, 0);

        return String(ip);
    }
    \end{lstlisting}
\end{figure}

Unfortunately this does not compile and raises an error due to the relayControl function.
According to the documentation this is because functions that are registered with the
cloud can only return int which raises a fundamental problem with this design. As such
I decided that I will register a variable cloud that will contain the ip address and can
be retrieved by a GET request. The working code is shown below.

\begin{figure}[H]
    \begin{lstlisting}
    int RELAY1 = D0;
    char ip[15];

    void setup()
    {
       //Initilize the relay control pins as output
       pinMode(RELAY1, OUTPUT);
       // Initialize all relays to an OFF state
       digitalWrite(RELAY1, LOW);

       //register the Particle function
       Particle.function("relay", relayControl);\
       Particle.variable("IpAddr", ip);
       Particle.subscribe("particle/device/ip" , handler);
       Particle.publish("particle/device/ip");

    }

    void loop()
    {
       // This loops for ever
    }

    void handler(const char *topic, const char *data)
    {
        //sprintf(ip, "%s", data);
        strcpy(ip, data);
    }

    // command format r1,HIGH
    int relayControl(String command)
    {
        digitalWrite(0, 1);
        delay(500);
        digitalWrite(0, 0);

        return 1;
    }
    \end{lstlisting}
    \caption{Public Ip Example code} \label{fig:publicIpExample}
    \vspace{0.5cm}
    The code in figure ~\ref{fig:publicIpExample} can be shown working in figure ~\ref{fig:ipTest}.
\end{figure}


\begin{figure}[H]
    \noindent\makebox[\textwidth]{\includegraphics[width=\paperwidth-3cm]{pictures/IpTest.png}};
    \caption{Ip test} \label{fig:ipTest}
    The ip address is shown in the red box and I have blacked out the access tokens for the request
    for obvious reasons.
\end{figure}

\subsection{The client interface}
This section will talk about how I plan to implement a client that can trigger the boards functions
remotely. Since this is the design section as oppose to the build I will only out line how I plan to
do this without actually coding it. The first and foremost thing that the interface requires is a
way to communicate with server. At the moment with the design of the board we can communicate with
POST and GET http requests meaning I need a way of performing those actions. Fortunately python has
many libraries that allow you to perform such actions. One of the most popular and the one that I
will be using for the interface is the requests library\cite{pythonRequests}. You can send get
requests using the "requests.get(<url>)" method where "<url>" is the url that you want to send the
request to. We will set this url to the one of the particle api.

To then connect to the server and communicate with it I can use the "pxssh" library and use that to
create a proper shell session between the server and the interface. I will go into details about
how I program this in interface section of the build section.


\pagebreak

\section{Build}

\section{Information}

\begin{thebibliography}{56}

\bibitem{raspberryPi}
    \begin{itemize}
        \item Information for the raspberry pi was gather at this website
        \begin{itemize}
            \item Article title: Maker Shed: Arduino | Raspberry Pi | 3D Printers | Microcontroller Kit
            \item Website title: Maker Shed
            \item URL          : \url{https://www.makershed.com/pages/raspberry-pi-comparison-chart}
            \item Date         : ???????
        \end{itemize}
    \end{itemize}

\bibitem{beagleboard}
    \begin{itemize}
        \item Information for the specs was found on this website
        \begin{itemize}
            \item Article title: Selecting BeagleBoard Hardware | DigiKey
            \item Website title: Digikey.com
            \item URL          : \url{https://www.digikey.com/en/product-highlight/t/texas-instruments/beagleboard}
            \item Date         : ???????
        \end{itemize}

        \item Information for the price of a begalbone kit
        \begin{itemize}
            \item Article title: BeagleBone Black Wireless Starter Kit
            \item Website title: Amazon.com
            \item URL: \url{https://www.amazon.com/BeagleBone-Black-Wireless-Starter-Kit/dp/B01NGTOX40/ref=sr_1_2?ie=UTF8&qid=1510164000&sr=8-2&keywords=beaglebone+wifi&dpID=51g7KiFRteL&preST=_SY300_QL70_&dpSrc=srch}
            \item Date         : ???????
        \end{itemize}

        \item Information for the spec specific to the BeagleBone black wireless
        \begin{itemize}
            \item Article title: BeagleBoard.org - black-wireless
            \item Website title: Beagleboard.org
            \item URL          : \url{https://beagleboard.org/black-wireless}
            \item Date         : ???????
        \end{itemize}

        \item Information for the spec specific to the BeagleBone black
        \begin{itemize}
            \item Article title: BeagleBoard.org - black
            \item Website title: BeagleBoard.org - black
            \item URL          : \url{https://beagleboard.org/black/}
            \item Date         : ???????
        \end{itemize}

    \end{itemize}

\bibitem{ardino}
    \begin{itemize}
        \item Information and specs for the arduino uno was gathered at this website
        \begin{itemize}
            \item Author       : Arduino
            \item Article title: Arduino Uno | Make: DIY Projects and Ideas for Makers
            \item Website title: Make: DIY Projects and Ideas for Makers
            \item URL          : \url{https://makezine.com/product-review/arduino-uno/}
            \item Date         : ???????
        \end{itemize}
    \end{itemize}

\bibitem{particlePhotonSpec}
    \begin{itemize}
        \item A product review for the particle photon
        \begin{itemize}
            \item Author       : Particle
            \item Article title: Particle Photon | Make: DIY Projects and Ideas for Makers
            \item Website title: Make: DIY Projects and Ideas for Makers
            \item URL          : \url{https://makezine.com/product-review/particle-photon/}
            \item Date         : ???????
        \end{itemize}

        \item A data sheet for the particle photon
        \begin{itemize}
            \item Author       : Particle
            \item Article title: Particle
            \item Website title: Docs.particle.io
            \item URL          : \url{https://makezine.com/product-review/particle-photon/}
            \item Date         : ???????
        \end{itemize}
    \end{itemize}

\bibitem{transistor}
    \begin{itemize}
        \item Information and specs for the Arduino Uno was gathered at this website
        \begin{itemize}
            \item Author       : SparkFun Kit
            \item Article title: Transistors - learn.sparkfun.com
            \item Website title: Learn.sparkfun.com
            \item URL          : \url{https://learn.sparkfun.com/tutorials/transistors}
            \item Date         : ???????
        \end{itemize}
    \end{itemize}

\bibitem{relaySheild}
    \begin{itemize}
        \item Information regarding the particle photon relay sheild
        \begin{itemize}
            \item Author       : Particle
            \item Article title: Particle
            \item Website title: Docs.particle.io
            \item URL          : \url{https://docs.particle.io/reference/firmware/photon/#get-public-ip}
            \item Date         : ???????
        \end{itemize}
    \end{itemize}

\bibitem{publicIPDocs}
    \begin{itemize}
        \item Information and specs for the Ardino Uno was gathered at this website
        \begin{itemize}
            \item Author       : Particle
            \item Article title: Particle
            \item Website title: Docs.particle.io
            \item URL          : \url{https://docs.particle.io/reference/firmware/photon/#get-public-ip}
            \item Date         : ???????
        \end{itemize}
    \end{itemize}

\bibitem{pythonRequests}
    \begin{itemize}
        \item Information and specs for the Ardino Uno was gathered at this website
        \begin{itemize}
            \item Article title: Requests: HTTP for Humans — Requests 2.18.4 documentation
            \item Website title: Docs.python-requests.org
            \item URL          : \url{http://docs.python-requests.org/en/master/}
            \item Date         : ???????
        \end{itemize}
    \end{itemize}
\end{thebibliography}

\subsection{Activity Log}

\begin{tabularx}{\textwidth}{| X | X |}
    \hline
    \textbf{Date}            &               \textbf{Activity}           \\ \hline
    21/08/17                 &This is the first day when I outlined what \\
                             &I want my project to be and decided on a   \\
                             &title. I also began to \\ \hline
    28/08/17                 &This is where I create and outline of what \\
                             &I want to allocate to                      \\ \hline
    04/09/17                 &I started to research the                  \\
                             &boards that I was thinking of              \\
                             &using.                                     \\ \hline
    02/10/17                 &Finished the research on the boards and    \\
                             &started to look into the different          \\
                             &electrical switches I can use.             \\ \hline
    08/10/17                 &This is the day that I received            \\
                             &the particle photon and performed          \\
                             &the transistor test with it.               \\ \hline
    06/11/17                 &Finished the research on electrical        \\
                             &switches                                   \\ \hline
    20/11/17                 &Started the design and looking into the    \\
                             &particle photon docs.                      \\ \hline

\end{tabularx}

\end{document}
